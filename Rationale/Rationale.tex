\documentclass[english,12pt,a4paper]{article}
\usepackage[T1]{fontenc}
\usepackage{babel}
\usepackage{indentfirst}
\title{Thesis Rationale}
\author{Bryan Jett T. Calulo, Lovely Angeline OL. Cunanan, Elijah Iñigo C. Fabian}
\begin{document}
	\maketitle
	
	\section*{Overview}
	The Philippines is highly vulnerable to natural disasters, so careful planning is essential for evacuating victims. Allocating shelters requires careful consideration, considering factors such as the distance from communities to shelters, the cost of maintaining the shelters, and the distance from workplaces in the area. Existing models that address these considerations can be solved using a genetic algorithm (GA), which necessitates the development of a computer program. This paper discusses these existing models and focuses on creating a computer program to solve them, followed by simulating the models in the Hagonoy area.
	
	The Philippines is highly vulnerable to natural disasters, making careful planning essential for evacuating victims. Allocating shelters requires careful consideration, taking into account factors such as the distance from communities to shelters, the cost of maintaining the shelters, and the distance from workplaces in the area. Existing models that address these considerations can be solved using a genetic algorithm (GA), which necessitates the development of a computer program. This paper discusses these existing models and focuses on creating a computer program to solve them, followed by simulating the models in the Hagonoy area.
	
	\section*{Rationale}
	The efficient allocation of shelters is essential and critical in disaster-prone areas. Effective evacuation planning can greatly impact the safety and well-being of the affected population. Traditional approaches to shelter allocation often result in overcrowding, inefficient resource use, and delayed response times. This thesis seeks to tackle these challenges by developing a computer program that utilizes a genetic algorithm to optimize shelter allocation based on key factors.
	
	
	This thesis aims to improve disaster preparedness by providing a tool that facilitates more efficient decision-making and optimizes shelter placements regarding accessibility, cost-efficiency, and community proximity. Once validated through simulations in Hagonoy, Bulacan, the proposed solution has the potential to significantly reduce the risks associated with natural disasters in that area and enhance the overall management of shelter allocation and emergency resources.
	
	\section*{Significance and Scope of the Study}
		\subsection*{Significance of the Study}
			The resulting models and program from this thesis will primarily benefit the population of the disaster-prone area of Hagonoy, Bulacan by optimizing allocation of shelters, thus leading to faster evacuation times and better distribution of resources for use in the welfare of its citizens. The resulting optimization of shelter locations also extends to the efficient use of financial and logistical resources, minimizing the costs required in the maintenance and operation of shelters, thus allowing reallocation of previously spent resources into other areas.
		\subsection*{Scope of the Study}
			This thesis on the optimization of the allocation of evacuation shelters is geographically limited to Hagonoy, Bulacan due to the municipality being particularly vulnerable to natural disasters, especially flooding; however, this study has the capability to be applied to other disaster-prone areas with some adjustments, which will not be covered in this thesis. The thesis will utilize real-world information from Hagonoy, including existing shelter locations and geographic layouts, as well as cost estimates for shelter maintenance and other operations; this data will be acquired by the researchers as well as being provided by the government of Hagonoy. This thesis is also operating under assumptions regarding human behavior, and the unpredictability of disaster impacts.
	
	
\end{document}