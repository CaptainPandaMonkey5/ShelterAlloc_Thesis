\section{Background of the Study}

The Philippines, situated within the Pacific Ring of Fire and the Pacific typhoon belt, is prone to typhoons, earthquakes, and volcanic eruptions.
These natural disasters often result in substantial loss of life, displacement of communities, and economic damage \parencite{1}. Despite ongoing efforts to enhance disaster preparedness, the insufficient allocation of shelters remains a critical challenge, leading to overcrowding, inefficient resource utilization, and heightened vulnerability for affected populations. The severe impact of natural disasters in the Philippines \parencite{1}. For example, Typhoon Haiyan, also known as Super Typhoon Yolanda, caused extensive devastation in 2013, underscoring the problems of inadequate disaster response and shelter allocation. The recurring nature of these events underscores the need to develop robust systems capable of meeting the high demand for shelter allocation.

Shelter location-allocation (or shelter allocation) refers to assigning displaced communities to available shelters during a disaster \parencite{2}. It is a crucial aspect of disaster prevention and mitigation that involves the safety and well-being of affected populations by providing secure, accessible, and adequate temporary shelter. Effective shelter allocation involves considerations such as shelter capacity, proximity to disaster sites, and the specific needs of vulnerable communities. These practices have evolved from traditional approaches to more systematic approaches incorporating modern technologies and methodologies, and it highlights the increasing importance of structured and efficient shelter allocation.

This study proposes using a genetic algorithm (GA), a computational method inspired by natural selection and evolution, to address the inefficiencies in shelter allocation. \textcite{3} discussed that these algorithms represent potential solutions to a given problem using a basic chromosome-like data structure and employ recombination operators to maintain essential information. Genetic algorithms are commonly regarded as tools for function optimization, mimicking the process of evolution by generating solutions to optimization problems through selection, crossover, and mutation. This algorithm has been applied to a wide variety of problems. In shelter allocation, genetic algorithms can optimize the assignment process by simultaneously considering factors such as shelter capacity, location, and accessibility. This approach aims to minimize overcrowding, improve resource distribution, and enhance overall efficiency in managing shelters during disasters.

Given the Philippines' frequent exposure to natural disasters, implementing a shelter allocation system is paramount. This study is dedicated to developing a system based on genetic algorithms to address the unique challenges experienced in the Philippines, focusing on Hagonoy, Bulacan, chosen for its high vulnerability to flooding and its challenges in managing shelter allocation during disasters. The proposed shelter allocation system has the potential to significantly improve disaster response efficiency, reduce overcrowding, and enhance safety for displaced individuals. Implementing a shelter allocation system using genetic algorithms in Hagonoy, Bulacan, is expected to improve disaster response significantly. By optimizing shelter allocation, the system can reduce overcrowding, improve access to resources, and enhance overall safety for displaced individuals. Additionally, the insights gained from this study could be applied to other municipalities facing similar challenges, contributing to the national effort to strengthen disaster resilience.


Cite your references. For instance,
\begin{enumerate}
	\item In-line or text citations
	\begin{itemize}
		\item According to \textcite{BooktagYear}, the . . .
		\item \textcite{JournaltagYear} showed that the . . .
		\item As discussed by \textcite{InbookTagYear}, the . . .
	\end{itemize}
	
	\item Post-paragraph or parenthetical citations
	\begin{itemize}
		\item The fact is . . . blah blah blah \parencite{WebsiteTag}.
		\item Result result result . . . blah blah blah \parencite{ProceedingsTag}.
		\item Statement of fact . . . blah blah blah \parencite{ThesisTag}
	\end{itemize}
\end{enumerate}
