\section{Background of the Study}

The background section sets the stage for the entire thesis by providing context and highlighting the importance of the study. Compose the Background of the Study here.

Cite your references. For instance,
\begin{enumerate}
	\item In-line or text citations
	\begin{itemize}
		\item According to \textcite{BooktagYear}, the . . .
		\item \textcite{JournaltagYear} showed that the . . .
		\item As discussed by \textcite{InbookTagYear}, the . . .
	\end{itemize}
	
	\item Post-paragraph or parenthetical citations
	\begin{itemize}
		\item The fact is . . . blah blah blah \parencite{WebsiteTag}.
		\item Result result result . . . blah blah blah \parencite{ProceedingsTag}.
		\item Statement of fact . . . blah blah blah \parencite{ThesisTag}
	\end{itemize}
\end{enumerate}
