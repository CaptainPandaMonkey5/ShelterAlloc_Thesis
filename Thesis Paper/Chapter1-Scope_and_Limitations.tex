\section{Scope and Limitations}

This thesis on the optimization of the allocation of evacuation shelters is geographically limited to Hagonoy, Bulacan due to the municipality being particularly vulnerable to natural disasters, especially flooding; however, this study has the capability to be applied to other disaster-prone areas with some adjustments, which will not be covered in this thesis. Moreover, this thesis will use the Bilevel No Transfer model derived from the General Single Level model (GEN) and Hierarchical model (HIER) through Genetic Algorithm only.


The thesis will utilize real-world information from Hagonoy, including existing shelter locations and geographic layouts, as well as cost estimates for shelter maintenance and other operations; this data will be acquired by the researchers as well as being provided by the government of Hagonoy. This thesis is also operating under assumptions regarding human behavior, and the unpredictability of disaster impacts.


The project is limited to a 10-month period wherein all stages of the thesis and consequent testing must be completed. Prolonged data collection and project maintenance beyond this period will not be covered by the researchers.
