\section{Significance of the Study}

In 2022, the Philippines was ranked as the most disaster-prone country on the World Risk Index. The nation faces a diverse array of natural disasters each year, including typhoons, earthquakes, and volcanic eruptions \parencite{Atwii2022} . The country's location on the "Pacific Ring of Fire" also makes it prone to frequent earthquakes and volcanic activity. Its geographical features make it more vulnerable to various disasters, such as tsunamis, rising sea levels, storm surges, landslides, floods, flash floods, and droughts.
 

This project seeks to improve disaster preparedness by providing a program and a model that facilitates more efficient decision-making and optimizes shelter placements regarding accessibility, cost-efficiency, and community proximity. Once validated through simulations in Hagonoy, Bulacan, the proposed solution has the potential to significantly reduce the risks associated with natural disasters in that area and enhance the overall management of shelter allocation and emergency resources.


The resulting models and program from this project will primarily benefit and protect the citizens of the disaster-prone area of Hagonoy, Bulacan by optimizing allocation of shelters, thus leading to faster evacuation times and better distribution of resources for use in the welfare of its citizens. The resulting optimization of shelter locations also extends to the efficient use of financial and logistical resources, minimizing the costs required in the maintenance and operation of shelters, thus allowing reallocation of previously spent resources into other areas.


By providing insights into optimized shelter locations and evacuee distribution, this project seeks to benefit the following parties:
\begin{description}
\item[Local Government.] The local government of the target municipality is a beneficiary of this project due to being responsible for disaster management and response within their jurisdiction. The administration of the affected municipalities’ LGU would achieve a system that will assist in the evacuation and protection of citizens and thus may divert their attention elsewhere into other areas.
\item[Provincial Disaster Risk Reduction and Management Council (PDRRMC).] The PDRRMC is the group responsible for coordination of disaster risk reduction and management at the provincial level. This project’s resulting findings may be a valuable tool for the PDRRMC to assist them in terms of optimal shelter locations and distribution of evacuees.
\item[Communities.] The local communities within Hagonoy are directly affected by disasters, especially floods and are thus important beneficiaries of this project. The community of Hagonoy would in theory achieve faster response times, shelters that are located in optimal positions that take into account their homes and workplaces, and more systematic procedures in evacuation.
\item[Future researchers.] Future researchers are a beneficiary due to this project being open to the public and thus, researchers and developers may derive their own thesis projects that are similar for use in different areas of the Philippines.
\end{description}
